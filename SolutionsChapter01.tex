\documentclass[]{article}
\usepackage{fontspec}
\usepackage{parskip}
\usepackage{graphicx}
\usepackage[a4paper]{geometry}
\usepackage{tabularx}
\usepackage{amsmath}

\begin{document}

%title
\begingroup  
	\raggedright
	\LARGE Art of Electronics\\
	\LARGE Exercise Solutions Chapter 1\\
\endgroup
	
\section*{About}
	This is my personal solutions to exercises from the book Art of Electronics Third Edition. 

\section*{Resistors}

\begin{enumerate}
	
	\item Assume a 5k and a 10k resistor.
	
	\begin{enumerate}
		
		\item The combined resistance if connecting these resistors in series is calculated by adding the resistances.
		
		\begin{equation*}
		R = R_{5k} + R_{10k} = 5k + 10k = 15k
		\end{equation*}
		
		\item The combined resistance if connecting these resistors in parallel can calculated in two different ways.
		
		\begin{equation*}
				R = \dfrac{ R_{5k} \times R_{10k} }{ R_{5k} + R_{10k} } = \dfrac{ 5k \times 1k }{ 5k + 10k } = 3.3k
		\end{equation*}
		
		\begin{equation*}
				R = \dfrac{ 1 }{ \dfrac{ 1 }{ R_{5k} } + \dfrac { 1 } { R_{10k} } }
				= \dfrac{ 1 }{ \dfrac{ 1 }{ 5k } + \dfrac { 1 } { 10k } } 
				= \dfrac {1k}{0.2 + 0.1}= 3.3k
		\end{equation*}
		
		
	
	
	\end{enumerate}
		
\end{enumerate}
	
\end{document}